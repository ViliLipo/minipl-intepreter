\documentclass[12pt,a4paper]{article}
\usepackage[T1]{fontenc}
\usepackage[utf8]{inputenc}
\renewcommand{\familydefault}{\sfdefault}
\usepackage{graphicx}
\usepackage{verbatim}
\usepackage{float}
\usepackage{tikz}
\usepackage{pgfplots}
\usepackage{listings}
\pgfplotsset{compat=1.16}
\restylefloat{table}
\graphicspath{{Images/}}



\author{Vili Lipo, 014814253}
\title{Compilers Project 2020: Mini-pl interpreter documentation}
\begin{document}
\maketitle
\newpage

\section{General view of the application}
\subsection{Architecture}

The architecture of my implementation of Mini-pl interpreter
follows closely the pipeline pattern presented on the lectures.
The interpreter uses a multi-pass construction, as all of the
parsing is done before semantical analysis, and all semantical
analysis is done before the interpreting.

At the lowest level we have the class Source, that is responsible
for reading a source file and giving characters from it one by one.

This is given as a constructor parameter to the class Scanner, that
does the lexical analysis of the characters, and forms tokens out
of them. The Scanner consists of a collection of routines that
try to form a token by iterating the source. When a next
token is asked from the scanner it screens out whitespace and
comments and then it iterates through these
routines until one of them returns a valid token. If no
valid token is produced the scanner returns an error token.

When the source reaches the end-of-file the scanner produces
end of file token.

The parser is a recursive descent parser that asks for the tokens
one by one from the scanner. Every construct in the language has
a own method for parsing it. These methods produce abstract syntax 
tree nodes.

The abstract syntax tree produced by the parser
is first decorated by the TypeCheckVisitor, that
checks for semantical errors and creates a symbol table
based on variable declarations.

The table created by TypeCheckVisitor is then used by
InterpretingVisitor that interprets the source file.



\subsection{Testing}

The classes Source, Scanner and Parser have quite comprehensive
unit tests written to test their core functionality. Those
can be found in the '/tests' folder.
Also the `tests/test.minipl' includes all the examples given
in the language specification.

\subsection{Shortcomings}

\subsection{Running the program}

On Linux-machines running this program should be very straight forward.
Open the directory where the archive was unzipped in a terminal.
Then running 
\begin{lstlisting}[language=Sh]
python main.py ./tests/test.minipl
\end{lstlisting}

\section{Specifying the interpretation}

\subsection{Mini-PL token patterns}

\begin{verbatim}
<integer> = <digit>*
<string_literal> = "<alnum>*"
<identifier> = ([a-z] | [A-Z])([a-z] | [A-Z] | _ | [0-9])*
<range> = \.\.
<keyword> = var | for | end | in | do | read 
<keyword> =  print | int | string | bool | assert
\end{verbatim}

\subsection{Context-free grammar}
\begin{verbatim}
<prog> = <stmnts>
<stmts> = <stmnt> ";" <stmnts>
<stmnts> = <epsilon> // {<epsilon>}
<stmnt> = "var" <var_ident> ":" <type> <assign_value>
<stmnt> = <var_ident> ":=" <expr>
<stmnt = "for" <var_ident> "in" <expr> ".." <expr> "do"
              <stmnts> "end" "for"
<stmnt> = "read" <var_ident> 
<stmnt> =  "print" <expr> // {"print"}
<stmnt> "assert" "(" <expr> ")"
<assign_value> = ":=" <expr>| <epsilon>
<expr> = <opnd> <op> <opnd>
<opnd> = <int>
<opnd> = <string>
<opnd> = <var_ident>
<opnd> = "(" <expr> ")"
<type> = "int"
<type> = "string”
<type> = "bool"
<var_ident> = <ident>
<reserved_keywords = "var" | "for" | "end" | "in" | "do" | "read" |
                    "print" | "int" | "string" | "bool" | "assert"
\end{verbatim}

\subsection{Abstract syntax tree}

\subsection{Error Handling}

When the scanner encounters an error it sends an error token to the parser.

The parser uses context sensitive lookahead with exception driven error
handling to recover from syntax errors. Because of the way how the error
handling is written statements that are not able to be parsed will not show up
in the AST built by the parser, as the statement routine sees that there is
another symbol next, that is in its first set.

Semantic errors are discovered by TypeCheckVisitor.
These errors are printed to the user and prevent the interpreting process.

\section{Work hour log}



\end{document}
